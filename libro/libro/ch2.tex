%ch2.tex
\chapter{Up to Homotopy is Good Enough}\label{gdenuf}

\begin{center}
\small{\em A log with nine holes\/}---\,old Turkish riddle for a man
\end{center}

\section{Introducing homotopy}
In a topological category, a pair of maps   $f,g : X
\rightarrow Y $ which agree on  $A \subseteq X$  is said to admit a
{\bf homotopy}\index{homotopy|bi} $H$ from $f$ to $g$ {\bf relative
to}\index{rel|bi} $A$ if there is a
map\index{homotopy!relative|bi}\index{relative!homotopy|bi}
    $$ X\times \I \stackrel{H}{\longrightarrow} Y : (x,t) \longmapsto H_{t}(x) $$
with
$H_t(a)=H(a,t) = f(a)=g(a)$ for all $a\in A$,
$H_0 = H(\ ,0) = f$, and $H_1 = H(\ ,1) = g$.
Then we write  $f\stackrel{H}\sim g \  (rel A)$.

If  $A = \emptyset$   or  $A$ is clear from the context (such as $A={\ast}$
for pointed spaces, {\em cf.} below), then we write   $f\stackrel{H}\sim g$,
or sometimes just $f\sim g$  and say that  $f$  and  $g$  are  {\bf
homotopic}\index{homotopic|bi}.

We can also think of $H$  as either of:
\begin{itemize}
\item a 1-parameter family of maps
$$\{H_{t} : X \longrightarrow Y \ \mid \ t \in [0,1]\ \}\ \  \mbox{with}
\ \   H _{0} = f \ \ \mbox{and} \ \   H _{1} = g\, ;$$
\item  a curve  $c_{H}$   from   $f$   to $g$ in the function space   $Y^{X}$
 of maps from  $X$  to  $Y$
    $$ c_{H} : [0,1] \longrightarrow Y^X : t \longmapsto H_{t}\,. $$
\end{itemize}

We call   $f$  {\bf nullhomotopic}\index{nullhomotopic|bi} or  {\bf
inessential}\index{inessential|bi} if
it is homotopic to a constant map. Intuitively, we picture   $H$  as a
continuous deformation of the {\em graph\/} of  $f$  into that of  $g.$
The following is an easy exercise.
\begin{proposition}
For all  $A \subseteq X$, $\sim (rel A)$  is an
equivalence relation on the set of maps from  $X$  to  $Y$  which
agree on $A$.
\end{proposition}
Maps in the same equivalence class of   $\sim (rel A)$  are
said to be {\bf homotopic $(rel A)$}.

\begin{table}
\begin{center}
\framebox[1.5in]{\begin{tabular}{c | c | c }
$n$ & $S^n$ & $R^n$ \\ \hline
1 & 1 & 1 \\
2 & 1 & 1 \\
3 & 1 & 1 \\
4 & 1 & $\infty$ \\
5 & 1 & 1 \\
6 & 1 & 1 \\
7 & 28 & 1 \\
8 & 2 & 1 \\
9 & 8 & 1 \\
10 & 6 & 1 \\
11 & 992 & 1 \\
12 & 1 & 1 \\
13 & 3 & 1 \\
14 & 2 & 1 \\
15 & 16256 & 1 \\
\end{tabular}}
\caption{Numbers of distinct differentiable structures on real $n$-space
and $n$-spheres}
\label{diffstruc}
\end{center}
\end{table}
